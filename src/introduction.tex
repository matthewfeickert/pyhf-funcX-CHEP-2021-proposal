\section{Introduction}\label{sec:introduction}

Researchers in High Energy Physics (HEP) and other fields are encouraged by their funding bodies to take advantage of the High Performance Computing (HPC) facilities constructed at various institutions.
These facilities include capable machines such as Theta at Argonne National Laboratory with 280,000 cores and 192 hardware-accelerated GPUs~\cite{ThetaANL}.
While powerful, these architectures do not easily support the Python compute model.
Users must construct batch jobs and submit them to a queue for execution when compute time is available.
The results are stored on the file system and must be stitched back together once all of the jobs have completed.
On many of these systems, Python tooling lags the current state of the art and configuring modern Python libraries to use HPCs can be a tedious task and require expertise.\\

In HEP a core component of analysis of data collected at the Large Hadron Collider (LHC) is performing statistical inference for binned models to extract physics information.
The statistical fitting tools used in HEP have traditionally been implemented in C++, but in recent years \pyhf{}~\cite{pyhf,pyhf_joss}, a pure-Python library with automatic differentiation and hardware acceleration, has grown in use for analysis related statistical inference problems.
The fitting of multiple different hypotheses for new physics signatures (signals) is a computational problem that lends itself easily to parallelization, but is hampered on HPC environments by the additional tooling overhead required, which can be very difficult to master.
Through use of \funcX{}~\cite{funcX_paper}, a pure-Python high performance function serving system designed to orchestrate scientific workloads across heterogeneous computing resources, \pyhf{} can be used as a highly scalable (fitting) function as a service (FaaS) on HPCs.
