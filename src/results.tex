\section{Scaling of likelihood fits}\label{sec:results}

\textbf{Placeholder for results section.}
Should also mention the setup of cluster actually used.\\

The output of execution of the 125 signal hypothesis patches from the published analysis of a search for electroweakinos with the ATLAS detector using the full Run-2 dataset of \(139~\ifb\) of \(\sqrt{s} = 13\,\text{TeV}\) proton-proton collision data~\cite{SUSY-2019-08} is shown in \Cref{lst:funcX_demo_output}.
By orchestrating \pyhf{} fits packaged as \funcX{} functions across 70 worker nodes \funcX{} is able to download the \pyhf{} pallet containing the signal patches from HEPData~\cite{ATLAS_SUSY_1Lbb_pallet}, start \funcX{} worker nodes, send patched workspaces to each worker, fit the workspace and return the results with a wall time of just over two minutes.
\TODO{CLARIFY AND REVISE THIS.}\\

\begin{listing}
 \inputminted{text}{src/code/funcX_demo_output.txt}
 \caption{A subset of the run output from the execution of fitting the 125 signal hypothesis patches for the published ATLAS analysis~\cite{SUSY-2019-08}.
 The wall time (\texttt{real}) shows the simultaneous fit orchestrated by \funcX{} is performed in just over two minutes.}
 \label{lst:funcX_demo_output}
\end{listing}
