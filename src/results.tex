\section{Scaling of likelihood fits}\label{sec:results}
%
% The output of execution of the 125 signal hypothesis patches from~\cite{SUSY-2019-08} is shown in \Cref{lst:funcX_demo_output}.
A clear example of the performance gains provided by deploying \pyhf{} on \funcX{} is the execution of the 125 signal hypothesis patches from~\cite{SUSY-2019-08}.
By orchestrating \pyhf{} fits packaged as \funcX{} functions across 85 worker nodes, \funcX{} is able to receive posted JSON serializations of the \pyhf{} pallet containing the background workspace and signal patches downloaded from HEPData~\cite{ATLAS_SUSY_1Lbb_pallet}, start \funcX{} worker nodes, send patched workspaces to each worker, fit the workspace and return the results with a wall time of under 2 minutes and 30 seconds.
As this wall time includes data transfer to and from RIVER and worker node orchestration time, the time required for inference alone is even smaller.
Example typical run output and performance can be seen in~\Cref{lst:funcX_demo_output}.
The timing results for~\cite{ATLAS_SUSY_1Lbb_pallet}, along with the results from additional analyses that have published probability models as \pyhf{} pallets on HEPData, over multiple trials are summarized in~\Cref{table:performance} and compared to the fit time for all patches on a single node.
All code used in these studies is publicly available on GitHub at Ref.~\cite{study_code}.

\begin{table}[htpb]
\centering
\caption{Fitting performance on RIVER for analyses for 10 trials. The uncertainty on the mean wall time corresponds to the standard deviation of the fit times. The number of worker nodes used is approximate as per-run reporting is not available.}
\label{table:performance}
\begin{tabular}{@{}lrrr@{}}
\toprule
       Analysis & Worker nodes & Mean wall time (sec) & Best wall time (sec) \\
\midrule
ATLAS SUSY 1Lbb &           85 &        $156.2\pm9.5$ &                  140 \\
ATLAS SUSY XXXX &           85 &                  ??? &                  ??? \\
\bottomrule
\end{tabular}
\end{table}


\begin{listing}
 \inputminted{text}{src/code/funcX_demo_output.txt}
 \caption{A subset of the run output from the execution of fitting the 125 signal hypothesis patches for the published ATLAS analysis~\cite{SUSY-2019-08}.
 The wall time (\texttt{real}) shows the simultaneous fit orchestrated by \funcX{} is performed in 2 minutes and 20 seconds.}
 \label{lst:funcX_demo_output}
\end{listing}
