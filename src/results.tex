\section{Scaling of likelihood fits}\label{sec:results}

\textbf{Placeholder for results section.}
Should also mention the setup of cluster actually used.\\

The output of execution of the 125 signal hypothesis patches for the published analysis \textit{Search for direct production of electroweakinos in final states with one lepton, missing transverse momentum and a Higgs boson decaying into two \(b\)-jets in \(pp\) collisions at \(\sqrt{s} = 13\,\text{TeV}\) with the ATLAS detector}~\cite{SUSY-2019-08} (ATLAS SUSY 1Lbb) is shown in \Cref{lst:funcX_demo_output}.
By orchestrating \pyhf{} fits packaged as \funcX{} functions across 70 worker nodes \funcX{} is able to download the \pyhf{} pallet containing the signal patches from HEPData, start \funcX{} worker nodes, send patched workspaces to each worker, fit the workspace and return the results with a wall time of just over two minutes.
\TODO{CLARIFY AND REVISE THIS.}\\
\TODO{Get DOI for 1Lbb pallet}

\begin{listing}
 \inputminted{text}{src/code/funcX_demo_output.txt}
 \caption{A subset of the run output from the execution of fitting the 125 signal hypothesis patches for the published ATLAS SUSY 1Lbb analysis.
 The wall time (\texttt{real}) shows the simultaneous fit orchestrated by \funcX{} is performed in just over two minutes.}
 \label{lst:funcX_demo_output}
\end{listing}
